\documentclass{beamer}

% This work is licensed under a CC0 1.0 Universal
% http://creativecommons.org/publicdomain/zero/1.0/
%
% This theme was created by Ekaterina Gerasimova <kittykat3756@gmail.com> for
% GNOME.Asia 2016.

\mode<presentation>
{
  \usecolortheme[RGB={248,195,0}]{structure}
  \usetheme{Frankfurt}
  \setbeamercolor{frametitle}{fg=black}
  \setbeamercolor{title}{fg=black}
  \definecolor{gnomeasia}{RGB}{81,28,57}
  \setbeamercolor{section in head/foot}{fg=white, bg=gnomeasia}
  \setbeamercolor{bibliography entry author}{fg=gnomeasia}
  \setbeamertemplate{navigation symbols}{}

  \setbeamercovered{transparent}
}

\pgfdeclareimage[height=1cm]{gnomeasia-logo}{gnomeasia-logo}
\logo{\pgfuseimage{gnomeasia-logo}}


\usepackage[english]{babel}
\usepackage{listings}
\lstset{basicstyle=\footnotesize}
% or whatever

\usepackage[latin1]{inputenc}
% or whatever

\usepackage{times}
\usepackage[T1]{fontenc}
% Or whatever. Note that the encoding and the font should match. If T1
% does not look nice, try deleting the line with the fontenc.

\title%[Short Paper Title] (optional, use only with long paper titles)
{Contribute to Free software}

%\subtitle
%{Include Only If Paper Has a Subtitle}

\author
{Ekaterina Gerasimova, \texttt{kat@gnome.org}}
% - Give the names in the same order as the appear in the paper.
% - Use the \inst{?} command only if the authors have different
%   affiliation.

\date%[GNOME.Asia 2016] % (optional, should be abbreviation of conference name)
{GNOME.Asia Summit 2016}
% - Not really informative to the audience, more for people (including
%   yourself) who are reading the slides online

%\subject{Theoretical Computer Science}
% This is only inserted into the PDF information catalog. Can be left
% out. 

\begin{document}

\begin{frame}
  \titlepage
\end{frame}

\section{Contribute to Free software}

\subsection{Why are you here?}

\begin{frame}{What is Free software?}
  \begin{itemize}
    \item
      You can use it in any way you wish
    \item
      You can fix it or change it
    \item
      You can share it with your friends and family
    \item
      You can make it better and then share it
  \end{itemize}
\end{frame}

\begin{frame}{Why do you want to contribute?}
% Ask the audience
\visible<2->{
  \begin{itemize}
    \item
      Learn new skills
    \item
      Experience a real working environment
    \item
      Work with people from all over the world
    \item
      See real people use what you've made
    \item
      Include it on your CV
    \item
      You believe in the philosophy
  \end{itemize}
  }
\end{frame}

\begin{frame}{How can you contribute?}
% Ask the audience
\visible<2->{
  \begin{itemize}
    \item
      Code
    \item
      Design
    \item
      Websites
    \item
      Engagement/promotion/marketing
    \item
      Document
    \item
      Accessibility
    \item
      Translate
    \item
      Test
    \item
      Educate
    \item
      Organise events
    \item
      Help others become contributors
  \end{itemize}
  }
%Attend events, join/run LUGs and special interests groups
\end{frame}


\section{Start contributing}

\subsection{How can you contribute?}

\begin{frame}[fragile,plain]{Get started!}
  \pgfdeclareimage[width=10cm]{newcomers}{newcomers}
  \center{\pgfuseimage{newcomers}}
\end{frame}

\begin{frame}[fragile,plain]{The release cycle}
  \pgfdeclareimage[width=11cm]{release-process}{release-process2}
  \center{\pgfuseimage{release-process}}
\end{frame}

\begin{frame}{Make a contribution}
  \pgfdeclareimage[height=4.5cm]{bad-flowchart}{bad-flowchart.pdf}
  \center{\pgfuseimage{bad-flowchart}}
  % If you think the first step towards contributing is sending in a patch, you're probably wrong.
\end{frame}

\begin{frame}{It's all about the people}
  \begin{itemize}
  \item
    Talk to people\ldots
    \begin{itemize}
    \item
      IRC
    \item
      Mailing lists
    \item
      Here and now
    \end{itemize}
  \item
    Use available resources
  \item
    Ask for help
  \end{itemize}
\end{frame}

\begin{frame}{Etiquette}
  \begin{itemize}
  \item
    Assume people mean well
  \item
    Try to be concise
  \item
    Be patient and generous
  \item
    Be respectful and considerate
  \end{itemize}
\end{frame}

\begin{frame}{Workflow}
  \begin{columns}
  \column{.45\textwidth}
    \begin{itemize}
    \item
      Find out about the project workflow and follow it
      \begin{itemize}
      \item
        \alert<2>{Bugzilla}
      \item
        \alert<3>{Mailing lists}
      \item
        \alert<4>{IRC}
      \item
        \alert<5>{Email}
      \item
        \alert<6-6>{Face-to-face meetings}
      \end{itemize}
    \item
      Build and test
    \end{itemize}
  \column{.55\textwidth}
  \only<1>{
    \pgfdeclareimage[height=4.3cm]{question}{question}
    \center{\pgfuseimage{question}}
    }
  \only<2>{
    \pgfdeclareimage[height=4.3cm]{bugs}{bugs}
    \center{\pgfuseimage{bugs}}
    }
  \only<3>{
    \pgfdeclareimage[height=4.3cm]{mail}{mail}
    \center{\pgfuseimage{mail}}
    }
  \only<4>{
    \pgfdeclareimage[height=4.3cm]{irc}{irc}
    \center{\pgfuseimage{irc}}
    }
  \only<5>{
    \pgfdeclareimage[height=4.3cm]{email}{email}
    \center{\pgfuseimage{email}}
    }
  \only<6>{
    \pgfdeclareimage[height=4.3cm]{guadec}{guadec}
    \center{\pgfuseimage{guadec}}
    }
  \end{columns}
\end{frame}

\begin{frame}{How things happen}
  \pgfdeclareimage[height=4.5cm]{good-flowchart}{good-flowchart.pdf}
  \center{\pgfuseimage{good-flowchart}}
\end{frame}

\begin{frame}{Followup}
  \begin{itemize}
  \item
    Be patient
  \item
    Follow up with the reviewer
  \item
    Follow up on the review
  \end{itemize}
\end{frame}

\begin{frame}{Make good contributions}
  \begin{itemize}
  \item
    Use the reviewer's time well
  \item
    Keep to the style of the project
  \item
    Follow the review process
  \item
    Respond in a timely manner
  \item
    Write good commit messages
  \end{itemize}
\end{frame}

\begin{frame}{Experience and growth}
  \begin{itemize}
  \item
    The more you contribute, the more you communicate with people
  \item
    The more you communicate, the more you become part of the community
  \item
    Once you are part of the community, you can become a Foundation member
  \end{itemize} 
  \begin{itemize}
  \item
    Help others?
  \end{itemize}
\end{frame}

\begin{frame}{Internships}
  \pgfdeclareimage[width=10cm]{gsoc}{gsoc.jpg}
  \center{\pgfuseimage{gsoc}}
  \pgfdeclareimage[width=10cm]{outreachy}{outreachy.png}
  \center{\pgfuseimage{outreachy}}
\end{frame}

\begin{frame}{Summary}
  % Keep the summary *very short*.
  \begin{itemize}
  \item
    The \alert{passion} for a FLOSS project comes from your personal interest in the project
  \item
    \alert{Collaborate} with others and \alert{learn together} to get ahead
  \item
    \alert{Explore} other projects
  \end{itemize}
  
  With special thanks to
  \begin{itemize}
  \item
    Andr\'{e} \v{C}. Klapper, \texttt{ak-47@gmx.net} 
  \item
    Sindhu S, \texttt{sindhus@gnome.org}
  \end{itemize}
\end{frame}

% All of the following is optional and typically not needed. 
\section<presentation>*{Resources}
\subsection<presentation>*{Resources}

\begin{frame}{Resources}
    
  \begin{thebibliography}{10}

  \bibitem{code}
    Start contributing:
    \newblock https://www.gnome.org/get-involved/
  \bibitem{code}
    Source code:
    \newblock https://git.gnome.org/
  \bibitem{code}
    Bugzilla:
    \newblock https://bugzilla.gnome.org/
  \bibitem{code}
    Mailing lists:
    \newblock https://mail.gnome.org
  \bibitem{code}
    IRC:
    \newblock https://wiki.gnome.org/IRC

  \end{thebibliography}
\end{frame}

\end{document}
